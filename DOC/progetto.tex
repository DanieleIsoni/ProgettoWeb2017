\documentclass[11pt,a4paper,twoside]{report} %impostazione della classe del codumento correttamente
\usepackage[english]{babel}%sillabazione
\usepackage[utf8]{inputenc}%accenti

%per stampare
%\usepackage[inner=4cm,	outer=1cm,		bottom=1.8cm,		top=1.8cm]	{geometry}

%per leggere su monitor
\usepackage[inner=2cm,	outer=2cm,		bottom=1.8cm,		top=1.8cm]	{geometry}


\usepackage[titles]{tocloft} %table of content

\usepackage{hyperref}%link alle varie sezioni e ai capitoli
\hypersetup{
    colorlinks,
    citecolor=black,
    filecolor=black,
    linkcolor=black,
    urlcolor=black
}

\usepackage[parfill]{parskip} %le nuov5e righe non iniziano indentate

\usepackage{caption}

\newcommand*\E{\`E\ } %comando \E per fare la e accentata maiuscola

%pacchetti non configurati direttamente
\usepackage{graphicx,wrapfig,epigraph,titlesec,lipsum,xcolor,anyfontsize}
\usepackage{bold-extra}



\graphicspath{{img/},{../}}
\usepackage{listings}
\usepackage{courier}
\usepackage{multicol}
\lstset{
	frame=single,
	basicstyle=\ttfamily
}


%stile di titoli di chapter section e subsection

\newcommand{\hsp}{\hspace{6pt}}

\titleformat{\chapter}[block]
	{\fontsize{35pt}{0pt}\sffamily\bfseries}
	{\fontsize{60pt}{0pt}\sffamily\thechapter\hsp\textcolor{gray75}{$|$}\hsp}
	{0pt}{\fontsize{30pt}{0pt}\bfseries}

\titlespacing{\chapter}{0pt}{-30pt}{30pt}


\usepackage{tikz}
\usepackage{epigraph,anyfontsize}


\renewcommand\epigraphflush{flushright}
\renewcommand\epigraphsize{\normalsize}
\setlength\epigraphwidth{0.7\textwidth}

\definecolor{titlepagecolor}{cmyk}{.10,.60,1,.20}

\definecolor{gray75}{gray}{0.75}

\makeatletter
\def\printauthor{%
	{\Large \sffamily\bfseries \@author}}
\def\printtitle{%
	{	\sffamily\bfseries \noindent

		\fontsize{30pt}{60pt} \selectfont \@title}
}
\def\printdate{%
	{ \sffamily \large \@date}}

\makeatother


% The following code is borrowed from: http://tex.stackexchange.com/a/86310/10898

\newcommand\titlepagedecoration{%
	\begin{tikzpicture}[remember picture,overlay,shorten >= -10pt]

	\coordinate (aux1) at ([yshift=-15pt]current page.north east);
	\coordinate (aux2) at ([yshift=-410pt]current page.north east);
	\coordinate (aux3) at ([xshift=-4.5cm]current page.north east);
	\coordinate (aux4) at ([yshift=-150pt]current page.north east);

	\begin{scope}[titlepagecolor!40,line width=12pt,rounded corners=12pt]
	\draw
	(aux1) -- coordinate (a)
	++(225:5) --
	++(-45:5.1) coordinate (b);
	\draw[shorten <= -10pt]
	(aux3) --
	(a) --
	(aux1);
	\draw[opacity=0.6,titlepagecolor,shorten <= -10pt]
	(b) --
	++(225:2.2) --
	++(-45:2.2);
	\end{scope}
	\draw[titlepagecolor,line width=8pt,rounded corners=8pt,shorten <= -10pt]
	(aux4) --
	++(225:0.8) --
	++(-45:0.8);
	\begin{scope}[titlepagecolor!70,line width=6pt,rounded corners=8pt]
	\draw[shorten <= -10pt]
	(aux2) --
	++(225:3) coordinate[pos=0.45] (c) --
	++(-45:3.1);
	\draw
	(aux2) --
	(c) --
	++(135:2.5) --
	++(45:2.5) --
	++(-45:2.5) coordinate[pos=0.3] (d);
	\draw
	(d) -- +(45:1);
	\end{scope}
	\end{tikzpicture}%
}

\renewcommand\maketitle{
	\begin{titlepage}

		\noindent
		\printtitle\par
		%\epigraph{ NOME }
		%{\textsc{Sviluppo di una Web Application in JSP}}
		\null\vfill
		\vspace*{1cm}
		\noindent
		\hfill
		\begin{minipage}{0.55\linewidth}
			\begin{flushright}
				\printauthor

				\vspace{10pt}

				\printdate
			\end{flushright}
		\end{minipage}
		%
		\begin{minipage}{0.02\linewidth}
			\rule{1pt}{65pt}
		\end{minipage}
		\titlepagedecoration
	\end{titlepage}

}














\author{Matteo Battilana - 180209\\Massimo Girondi - 178114\\Daniele Isoni - 181839}

\title{Progetto di \\Introduzione alla \\programmazione per il Web}
\date{Versione del \today}
\begin{document}
\maketitle
\clearpage\null\newpage
\tableofcontents
\clearpage\null\newpage

\chapter{Tecnologie utilizzate}
\lipsum
\chapter{Database}

\chapter{Librarie incluse}


\chapter{Autocompletamento}

\section{Barra di ricerca}
L'autocompletamento presente nella barra di ricerca è stato realizzato mediante la libreria JavaScript  \textbf{bootstrap-ajax-typeahead}, la quale recupera i suggerimenti per l'autocompletamento dalla servlet AutoCompleteServlet mediante richieste Ajax, a cui la servlet rispone in formato JSON.

La servlet effettua una ricerca per similarità (secondo l'algoritmo di Jaro-Winkler) all'interno di una lista, la quale viene caricata all'avvio dell'applicazione analizzando i nomi dei prodotti presenti nel database e aggiornata a intervalli prestabiliti (ogni 30 minuti) mediante le classi Executors e ScheduledExecutorService presenti all'interno di Java stesso.

Abbiamo scelto questa implementazione per evitare di sovraccaricare il database eseguendo una query a ogni richiesta di autocompletamento (e quindi a ogni tasto premuto).
Un'altra opzione, poi scartata, era aggiornare la lista dell'autocompletamento a ogni modifica della tabella dei prodotti, ma questo avrebbe rallentato ogni operazione sulla tabella stessa.

L'algoritmo di Jaro-Winkler è utilizzato anche durante la scansione del database, per evitare di inserire nella lista dei suggerimenti titoli doppi o troppo simili (ad esempio ``frigo'' e ``frigorifero'') dato che questi verrebbero comunque selezionati al momento della ricerca, dove è applicata ancora una volta la similarità.

\section{Indirizzo}
L'autocompletamento presente nel form di creazione di un nuovo negozio per l'indirizzo è stato utilizzato mediante la libreria JavaScript \textbf{Google Maps API}. Queste API permettono tramite chiamate asincrone di ricevere i suggerimenti i quali vengono mostrati mediante un dropdown. Una volta fatta la scelta vengono riempiti i campi del bean con indirizzo completo, e coordinate (latitudine e longitudine).

\chapter{Risultati della ricerca}
la ricerca tra i prodotti è realizzata mediante una servlet che elabora la richiesta e restituisce la lista dei prodotti sottoforma di JSON. L'unico parametro obbligatorio è \texttt{q} che rappresenta la query di ricerca.
I parametri opzionali sono:
\begin{itemize}
  \item \texttt{p}: la pagina desiderata (nel caso di  una ricerca multipagina)
  \item \texttt{s}: il metodo di ordinamento dei risultati, di default alfabetico (0), alfagetico inverso(1), prezzo crescente (2), prezzo decrescente(3), valutazione decrescente(4),  numero recensioni decrescente(5).
  \item \texttt{c}: la categoria
  \item \texttt{min} e \texttt{max}: gli estremi dei prezzi.
  \item \texttt{minRev}: il minimo del valore delle recensioni.
\end{itemize}
Il controllo sul nome del prodotto e sul range di prezzo è effettuato all'interno del DAO del prodotto (mediante SQL per il prezzo e algoritmo di Jaro-Winkler per il nome), mentre il controllo sulla categoria e sul valore minimo della media delle recensioni è effettuato mediante un predicato invocato dal metodo \texttt{removeIf} della lista prodotti.
Oltre alla lista prodotti nel risultato sono restituiti anche il numero di pagine, il tipo di ordinamento usato e la pagina corrente.


\chapter{Logging}
Il logging, fondamentale per un'applicazione di questo tipo, è stato implementato mediante la libreria Apache Log4j, che consente di gestire in modo molto semplice i vari livelli di log e il salvataggio automatico sul disco.

Abbiamo poi creato un wrapper attorno alla libreria per limitare il più possibile il numero di istruzioni richieste per scrivere un messaggio nel log.

I log vengono salvati nella cartella \texttt{\$CATALINA\_HOME/logs} in file testuali che iniziano con il nome \texttt{BuyHub}.
La cartella \texttt{\$CATALINA\_HOME} è in genere la cartella di installazione di TomCat o GlassFish, ma potrebbe variare in base alla configurazione del sistema.

\chapter{Dettagli implementativi}
\section{Orari di apertura e negozi solo online}
Per semplificare la struttura del DB abbiamo deciso di implementare i vari punti vendita dei negozi come una relazione uno a molti tra \texttt{shop} e \texttt{coordinate}. All'interno della tabella \texttt{coordinate} abbiamo inserito il campo \texttt{opening\_hours} che contiene gli orari del punto vendita, nel caso in cui per il cliente sia disponibile il ritiro in negozio. Nel caso di negozi che effettuano soltanto vendita online questo campo resta vuoto e viene sostituito all'interno della pagina web con un messaggio che specifica le modalità di vendita solo online del negozio.

\chapter{Upload/Download}

L'upload dei file è stato gestito mediante la libreria \texttt{com.oreilly.servlet}, mentre il salvataggio dei file è gestito da alcuni metodi all'interno della classe \texttt{Utility}.

In particolare, il metodo \texttt{saveJPEG} si occupa di salvare un immagine, dandogli un nome casuale basato su un UUID, convertendola a JPEG, mediante la libreria ImageIO, integrata in Java.
Purtroppo questa libreria non gestisce correttamente le trasparenze nelle immagini PNG, e al momento, non è ancora stata trovata una soluzione per ovviare a questo problema, in quanto non è possibile dedurre se questo errore di codifica è avvenuto o no.

I file vengono poi salvati in una cartella definita all'interno del file \texttt{config.properties}, nel quale è possibile inserire sia un path relativo (in base alla propria cartella \texttt{catalina.base}), oppure un percorso assoluto, che verrà automaticamente interpretato e utilizzato per il salvataggio.

Il dowload avviene mediante la servlet \texttt{UploadedContentServlet}, che risponde alle richieste che arrivano alle url del tipo \texttt{/UploadedContent/*}, in modo tale da simulare a tutti gli effeti una cartella. Il file richiesto viene inviato in modalità ``inline'', e con il relativo MIME impostato. Nel caso il file non sia presente nel disco, viene restituita un'immagine che indica la non presenza del file.

La scelta di utilizzare questo sistema è stata piuttosto ardua, in quanto, non volendo usare path assoluti, l'unica opzione per caricare un file direttamente con Java, era caricarlo nella cartella di esecuzione (la \texttt{contextPath}). Nel momento in cui però sarebbe avvenuto un redeploy tutti i file sarebbero stati cancellati, oltre al fatto che alcuni server Java non permettono la scrittura nella \texttt{contextPath}. Altri server, invece, mantengono direttamente tutto il \texttt{.war} dell'applicazione in RAM, senza creare una cartella nel disco rigido, e creare cartelle locali alla \texttt{contextPath} in questo contesto è abbastanza sconsigliato (aumenterebbe a dismisura l'uso di RAM).

Un'altra soluzione, volendo realizzare un prodotto per il mondo reale, sono i CDN, in particolare quelli specifici per le immagini, come cloudinary.com o imgix.com. Questi servizi, però, sono in genere a pagamento e non permettono il controllo completo delle proprie immagini. Per questi motivi la scelta è ricaduta nel caricare le immagini in un path locale alla cartella di esecuzione, ma esterno al \texttt{contextPath}, in modo da preservare i file tra le diverse esecuzioni e deploy dell'applicazione.




\chapter{Verifica dell'account}

Quando un utente si registra al sito, è prevista una conferma della registrazione attraverso la mail, onde evitare account fasulli. All'interno della mail viene inserito un link che contiene un codice speciale che abilita l'utente all'accesso.

In particolare, tale codice è composto dall'ID utente e dall'Hash della password, grazie ai quali è possibile identificare l'utente e, data la bassa probabilità di indovinare l'MD5, rende difficile un eventuale attacco.

Questo codice è poi cifrato mediante AES128 secondo una chiave segreta presente nel server, che ci consente quindi di essere \textit{praticamente} certi che un utente smaliziato riesca ad accedere a un account non proprio. Per poter inviare correttamente il risultato mediante una URL il codice è codificato prima in Base64 e poi attraverso la funzione \texttt{URLEncode} di Java. Il codice è quindi così formato:
\[
    URLEncode( Base64( AES( \mbox{id}\$MD5(\mbox{password}))))
\]


\chapter{Altri capitoli}
\chapter{Conclusioni}
\end{document}
